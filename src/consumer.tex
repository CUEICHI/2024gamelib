\section{コンシューマゲームの保存}

コンシューマゲームはいわゆるテレビゲームと呼ばれるものであり,
家庭のテレビに接続して利用することを想定したゲーム機や,
本体に液晶画面を搭載することで,完結したゲーム機として動くことが
想定されているものである.

コンシューマゲーム機はTVやモニターに接続して利用するコンソールゲーム機と
小型の画面を搭載した本体のみで動作する携帯ゲーム機に分類できる.
現在では,ニンテンドースイッチのように本体に液晶画面を搭載するがTV画面と
接続することも想定したハイブリッド型のゲーム機も存在している.

これらのゲーム機のうち初期のものはソフトウェアが内蔵されており,
いくつかの決められたゲームをプレイするために作られている.
その後,カートリッジ形式でROMを入れ替えることにより一つの筐体で
複数のゲームを遊ぶことが可能となっている.
その後,ROMカートリッジではなく,DVDやCDなどのメディアを利用した
ゲーム機も発売され,これらはゲームのみならずDVDやCDの再生機としての
利用が可能となっている.
現在では,インターネットの普及と拡大により
コンテンツのダウンロード販売が行われるようになったことにより,
物理メディアとしてのゲームソフトウェアを読み込まずに,
オンラインから本体にあるSSDのような記録メディアに
ソフトウェアを取り込むことが行われている.
この場合にはソフトウェアは物理的には存在しない点が
保存においては問題となると考えられる.

\subsection{コンソールゲームの問題}
多くのコンソールゲーム機は大量に生産,販売されており状態を問わない場合には
現在も本体を収集することは比較的容易であると考えられる.

\subsection{携帯型ゲームの問題}
携帯型ゲームにおいては,コンソールゲームの問題点に加え,
その本体の保存性に大きな問題が生じる.
専用の液晶画面を利用したゲーム機の場合にはその劣化により,
ゲームが行えなくなることがある.
また,操作のためのボタンやスティックが本体と一体化していることもあり,
故障などの問題も多く生じる.




