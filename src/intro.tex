\section{はじめに}
\label{sec:intro}

コンピュータ技術を利用した遊びであるコンピュータゲームは,汎用コンピュータのプログラムから始まり,
家庭用のゲームコンソールであるコンシューマゲームや,
業務用の専用筐体であるアーケードゲームなどへと発展している.
現在では,コンピュータゲームのプラットフォームとしては
専用のゲーム機のほか,パーソナルコンピュータやスマートフォンが
広く利用されている.

アーケードゲームはゲームセンターなどの娯楽施設に設置されることが主であり,
ゲームに飽きられるなどの理由により収益が下がったり,
経年劣化による故障により廃棄されることが多く,長期間維持されるものは
長期にわたって人気のあるゲームのみである.

コンソールゲーム機は家庭用に販売されているゲーム機であり,こちらも
経年によりゲームの提供がおこなわれなくなることでその需要がなくなる.
過去にはゲーム機のユーザは主に子供であったためその扱いも乱雑であり,
よい状態で保存されているゲーム機は少ないと考えられる.

また,コンソールゲーム機のソフトウェアはカートリッジやCD/DVDのような
形態で提供されるがこちらも人気のものは数百万本のレベルで出荷されているが,
人気のなかったものは1万本を切るようなものも多く存在する.

ゲームソフトウェアに関しては図書のように国会図書館への納本制度のような
公的機関による収集が行われておらず,すべてを網羅して保存することは困難である.

現在のスマートフォンの上で動作するゲームの多くはソーシャルゲームと呼ばれ
ゲームの機能のうちの多くをサーバ上で管理している.
そのため端末にダウンロードしたソフトウェアだけではそのゲームを起動することは
できず,サーバ側の運営が終了した時点でそのゲームを再現することが
できない.

このように,ゲームの保存においては書籍はもちろん,音楽や映画やTV番組のような
メディアコンテンツとも違った困難が伴うこととなる.

本稿ではゲームの種類による保存の困難さについてまとめるものである.