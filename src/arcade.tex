\section{アーケードゲームの保存}
\label{sec:arcade}

アーケードゲームはコンシューマーゲームと異なり,
ゲームセンターなどの娯楽施設において,
一回あたりの料金を課金してゲームを行うためのゲーム機である.
アーケードゲームは基本的に商業行為を行うための機材であり,
その経済的な合理性が消失した場合にはその筐体は転売あるいは破棄されるのが
一般的である.

アーケードゲームの種類としては汎用の筐体として,
アップライト型,テーブル型,ミディ型の3種類に分けられる.
アップライト型の筐体は立ち姿勢でゲームを行うものである.
テーブル型はモニタ画面が床と水平になったテーブルトップが
画面となった筐体である.
ミディ型はその中間であり,バーカウンターのようなところに設置する
スロットマシンのような形状をとっている.
これらの筐体は画面とコントローラとなるジョイスティックやボタンを共通とし,
ゲームの基板を入れ替えることでさまざまなゲームに対応することができる.


このような汎用ではなく専用筐体を用いるゲームはライド型,コントローラ型,
カードゲーム型に分けられる.

