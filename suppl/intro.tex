\section{はじめに}

情報処理学会では,基幹論文誌として論文誌ジャーナルの発行を行っている.
現在論文誌ジャーナル編集委員会では,
論文誌ジャーナルの論文掲載時のフォーマットとして
A4縦型2段組を採用している.
また,以前は投稿時と掲載時の形式が異なっていたが,
現在では,
投稿時も掲載時と同様のA4縦型2段組で受け付けることにした.



本稿では,
まずスタイルファイルを用いた論文のフォーマットに関して述べる.
新たなスタイルファイルでは,
極力特別なコマンドは使わずに,標準的な \LaTeX のスタイルを踏襲している.
論文フォーマットに関しては,\ref{sec:format}~章で
後述する指針に従って頂くが,
そこに規定されていること以外は標準的な\LaTeX のコマンドをそのまま使うことができる.
本稿は,そのスタイルファイルを実際に使っているので,
論文執筆の際に参考にされたい.




\footnotetext{本文は実際には論文誌ジャーナル編集委員会で作成したものである.}

また,論文誌ジャーナル編集委員会では,論文の執筆する際に,
著者がするべきこと,するべきでないことを「べからず集」としてまとめた.
本稿の後半に,論文の内容に関する指針になるように,
「べからず集」の内容をチェックリストとしてつけているので,
投稿する前の内容のチェックに利用されたい.


