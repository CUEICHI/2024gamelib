\section{投稿の流れ}


%2.1
\subsection{準備}


情報処理学会論文誌ジャーナルの \LaTeX スタイルファイルを含む
論文執筆キットは
\begin{quote}
\small
\|http://www.ipsj.or.jp/jip/submit/style.html|
\end{quote}
からダウンロードすることができる.論文執筆キットは以下のファイルを含んでいる.


\begin{Enumerate}
\item \|ipsj.cls      |: 原稿用スタイルファイル
\item \|ipsjpref.sty  |: 序文用スタイル
\item \|jsample.tex   |: 本稿のソースファイル
\item \|esample.tex   |: 英文サンプルのソースファイル
\item \|ipsjsort.bst  |: jBibTEX スタイル(著者名順)
\item \|ipsjunsrt.bst |: jBibTEX スタイル(出現順)
\item \|bibsample.bib |: 文献リストのサンプル
\item \|ebibsample.bib|: 英文文献リストのサンプル
\item \makebox[9.47zw][l]{{\tt ipsjtech.sty}}: 研究報告用スタイル
\item \| tech-jsample.tex|: 研究報告(和文)のサンプル
\item \| tech-esample.tex|: 研究報告(英文)のサンプル
\end{Enumerate}%
実行環境としては\LaTeXe を前提としているので,準備されたい.


Microsoft Wordに関しては,投稿されたフォーマットを基に,
業者が \LaTeX に変換して組版を行うので,
あくまでも参考としてしか使わないことを承知して頂きたい.



%2.2
\subsection{最終原稿の作成と投稿}

本稿に従って用意した原稿の \LaTeX ソースからpdfファイルを作成し,
Adobeのpdf readerで読めることを確認した後,
\begin{quote}
\small
\|https://mc.manuscriptcentral.com/ipsj|
\end{quote}
上記のURLから投稿する.
投稿の流れについては,
\begin{quote}
\small
\|http://www.ipsj.or.jp/journal/submit/manual/|\\
\|j_manual.html|
\end{quote}
を参照されたい.



なお,情報処理学会論文誌ジャーナルでは,
論文の著者が査読者の名前を知ることがないシングルブラインドの査読を取り入れている.




%2.3
\subsection{最終原稿の作成とファイルの送付}

投稿した論文の採録が決定したら,
査読者からのコメントなどにしたがって原稿を修正し,
図表などのレイアウトも最終的なものとする.
なお後の校正の手間を最小にするために,
この段階で記述の誤りなどを完全に除去するように綿密にチェックして頂きたい.



学会へは{\bf \LaTeX ファイル(をまとめたもの)}を送信する.
送信するファイル群の標準的な構成は \|.tex| と \|.bbl| であり,
この他にPostScriptファイルや特別なスタイルファイルがあれば付加する.
なお \|.tex| は印刷業者が修正することがあるので,
{必ず一つのファイルにする}.
また必要なファイルが全てそろっていること,
特に特別なスタイルファイルに洩れがないことを,注意深く確認して頂きたい.


ファイルの送信方法などについては,
採録通知とともに学会事務局から送られる指示に従う.




%2.4
\subsection{著者校正・組版・出版}


学会では用語や用字を一定の基準(常用漢字および
「現代仮名遣い」等)に従って修正することがある.
また \LaTeX の実行環境の差異などによって著者が作成した最終PDFと
実際の組版結果が微妙に異なることがある.
これらの修正や差異が問題ないかを最終的に確認するために,
著者にPDFファイルが送られるので,
もし問題があれば朱書によって指摘して送信する.
なお{\bf この段階での記述誤りの修正は原則として認められない}ので,
原稿送信時に細心の注意を払っていただきたい.


その後,著者の校正に基づき最終的な組版を行ない,
オンライン出版する.


