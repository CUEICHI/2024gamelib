\section{論文フォーマットの指針}
\label{sec:format}

以下,
情報処理学会論文誌ジャーナル用スタイルファイルを用いた論文フォーマットの
指針について述べるので,
これに従って原稿を用意頂きたい.\LaTeX を用いた
一般的な文章作成技術については,\cite{okumura, companion} 等を参考にされたい.



%4
\section{論文の構成}
\label{config}

ファイルは次のようになる.
下線部は投稿時に省略可能なもの.
また論文誌トランザクション特有コマンドについては \ref{sig}~節を参照されたい.

\noindent
\|\documentclass[submit]{ipsj}|\\
\quad 必要ならばオプションのスタイルを追加\\
\Underline{\|\setcounter{|{\bf 巻数}\|}{<巻数>}|}\\
\Underline{\|\setcounter{|{\bf 号数}\|}{<号数>}|}\\
\Underline{\|\setcounter{|{\bf page}\|}{<先頭ページ>}|}\\
\Underline{\|\|{\bf 受付}\|{<年>}{<月>}{<日>}|}\\
\Underline{\|\|{\bf 採録}\|{<年>}{<月>}{<日>}|}\\
\quad 必要ならばユーザのマクロをここに記述\\
\|\begin{document}|\\
\|\title{表題(和文)}|\\
\|\etitle{表題(英文)}|\\
\Underline{\|\affiliate{所属ラベル}{<和文所属>\\<英文所属>}|}\\
\quad 必要ならば \|\paffiliate| により現在の所属を宣言する\\
\Underline{\|\paffiliate{現所属ラベル}{<和現所属>\\<英現所属>}|}\\\\
\Underline{\|\author{情報 太郎}{Taro Joho}|}\\
\Underline{\|          {<所属ラベル>}[E-mail]|}\\
\Underline{\|\author{処理 花子}{Hanako Shori}|}\\
\Underline{\|          {<所属ラベル2,現所属ラベル3>}|}\\\\
\|\begin{abstract}|\\
\|<概要(和文)>|\\
\|\end{abstract}|\\
\|\begin{jkeyword}|\\
\|<キーワード>|\\
\|\end{jkeyword}|\\
\|\begin{eabstract}|\\
\|<概要(英文)>|\\
\|\end{eabstract}|
\|\begin{ekeyword}|\\
\|<KeyWords>|\\
\|\end{ekeyword}|\\
\|\maketitle|\\
\|\section{|第1節の表題\|}|\\
\dots\dots\dots\dots\dots\\
\quad \|<本文>|\\
\dots\dots\dots\dots\dots\\
謝辞がある場合は\\
\|\begin{acknowledgment}|\\
\|\end{acknowledgment}|\\\\
\|\begin{thebibliography}{99}%9 or 99|\\
\|\bibitem{1}|\\
\|\bibitem{2}|\\
\|\end{thebibliography}|\\\\
付録がある場合は\\
\|\appendix|\\
\|\section{|付録1節の表題\|}|\\\\
\Underline{\|\begin{biography}|}\\
\Underline{\|\profile{<X>}{<苗字 名前>}{<プロフィール文章>}|}\\
\Underline{\|\end{biography}|}\\
\|\end{document}|



%4.1
\subsection{オプション・スタイル}
\label{option}
\|\documentclass{ipsj}|のオプション\footnote{論文誌トランザクション用オプションは \ref{sig}~節で説明する.}として,
以下のものを用意してある.
{\bf 何も定義しなければ和文論文用の標準スタイル}となるが,
今回,組版の際に和文論文のタイトル,
和文論文種別に「{\bf 太ミン}」「{\bf 太ゴ}」のフォントを使用しているため,
\TeX 標準フォントに置き換える \|submit| というオプションを用意した.

\begin{enumerate}
\item\|submit         | フォント置換用
\item\|invited        | 招待論文
\item\|sigrecommended | 推薦論文
\item\|technote       | テクニカルノート用
\item\|preface        | 序文用
\item\|JIP            | 英文用
\end{enumerate}
これらのオプションは任意の組合せで使用が可能である.



なお,\|\usepackage| で補助的なスタイルファイルを指定した場合には,
最終原稿用のファイル群に必ずスタイルファイルを含める.
ただし,\LaTeXe の標準配布に含まれているもの
(たとえば \|graphicx|)については同封の必要はない.

スタイルファイルによっては論文誌スタイルと矛盾するようなものもあるので,
注意して使用して頂きたい.



%4.1.1
\subsubsection{研究報告専用オプション・スタイル}
\label{4-1-1}

上記オプションとは別に,研究報告専用のオプションを用意した.
\begin{enumerate}
\item\|techrep   | 研究報告(必須)
\item\|noauthor  | 英文著者表記無しの指定(和文;任意)
\end{enumerate}

和文の研究報告では,
和文キーワード,
英文著者名,
英文タイトル,
英文アブスト,
英文キーワードが任意入力となるため,
\|techrep|オプションを使用していれば,
任意項目が無くとも
コンパイルが止まることはない(\|tech-jsample.tex|参照).

\|\documentclass[submit,techrep]{ipsj}|\\
とすれば,研究報告のスタイルとなり,

\|\documentclass[submit,techrep,noauthor]{ipsj}|\\
とすれば,
英文著者名等が入らない研究報告のスタイルとなる.



英文の研究報告では,
キーワードのみが任意入力となるため,
\|noauthor|は使用できないので注意する
(\|tech-esample.tex|参照).


%4.2
\subsection{表題・著者名等}

表題,著者名とその所属,
および概要を前述のコマンドや環境により{\bf 和文と英文の双方について}定義した後,
\|\maketitle| によって出力する.



%4.2.1
\subsubsection{表題}

表題は,\|\title| および \|\etitle| で定義した表題はセンタリングされる.
文字数の多いものについては,適宜 \|\\| を挿入して改行する.

%4.2.2
\subsubsection{著者名・所属}

各著者の所属を第一著者から順に \|\affiliate| を用いてラベル(第1引数)を付けながら定義すると,
脚注に番号を付けて所属が出力される.
なお,複数の著者が同じ所属である場合には,一度定義するだけで良い.



現在の所属は \|\paffiliate| を用い,同様にラベル,所属先を記述する.
所属先には自動で「現在」,
\|\\|の改行で「Presently with」が挿入される.
著者名は \|\author| で定義する.
各著者名の直後に,英文著者名,所属ラベルとメールアドレスを記入する.
著者が複数の場合は \|\author| を繰り返すことで,
2人,3人,\dots と増えていく.
現在の所属や,複数の所属先を追加する場合には,
所属ラベルをカンマで区切り,追加すればよい.



また,
メールアドレス部分は省略が可能だが,必ず代表者のアドレスは必要となる.
なお,和文著者名,英文著者名は,姓と名を半角(ASCII)の空白で区切る.



%4.2.3
\subsubsection{概要}

和文の概要は \|abstract| 環境の中に,
英文の概要は \|eabstract| 環境の中に,それぞれ記述する.

%4.2.4
\subsubsection{キーワード}

和文の概要は \|jkeyword| 環境の中に,
英文の概要は \|ekeyword| 環境の中に,それぞれ1~5語記述する.

%4.3
\subsection{本文}

%4.3.1
\subsubsection{見出し}

節や小節の見出しには \|\section|, \|\subsection|, \|\subsubsection|,
\|\paragraph| といったコマンドを使用する.

\<「定義」,「定理」などについては,\|\newtheorem|で適宜環境を宣言し,
その環境を用いて記述する.


%4.3.2
\subsubsection{行送り}

2段組を採用しており,
左右の段で行の基準線の位置が一致することを原則としている.
また,節見出しなど,行の間隔を他よりたくさんとった方が読みやすい場所では,
この原則を守るようにスタイルファイルが自動的にスペースを挿入する.
したがって本文中では \|\vspace| や \|\vskip| を用いたスペースの調整を行なわないようにすること.




%4.3.3
\subsubsection{フォントサイズ}

フォントサイズは,
スタイルファイルによって自動的に設定されるため,
基本的には著者が自分でフォントサイズを変更する必要はない.



%4.3.4
\subsubsection{句読点}

句点には全角の「.」,読点には全角の「,」を用いる.
ただし英文中や数式中で「.」や「,」を使う場合には,
半角文字を使う.
「。」や「、」は使わない.

%4.3.5
\subsubsection{全角文字と半角文字}

全角文字と半角文字の両方にある文字は次のように使い分ける.

\begin{enumerate}
\item 括弧は全角の「(」と「)」を用いる.但し,英文の概要,図表見出し,
書誌データでは半角の「(」と「)」を用いる.

\item 英数字,空白,記号類は半角文字を用いる.ただし,句読点に関しては,
前項で述べたような例外がある.

\item カタカナは全角文字を用いる.

\item 引用符では開きと閉じを区別する.
開きには \|``| を用い,閉じには\|''| を用いる.
\end{enumerate}


%4.3.6
\subsubsection{箇条書}

箇条書に関する形式を特に定めていない.
場合に応じて標準的な \|enumerate|,
\|itemize|, \|description| の環境を用いてよい.



%4.3.7
\subsubsection{脚注}

脚注は \|\footnote| コマンドを使って書くと,
ページ単位に\footnote{脚注の例.}や\footnote{二つめの脚注.}のような
参照記号とともに脚注が生成される.
なお,ページ内に複数の脚注がある場合,
参照記号は \LaTeX を2回実行しないと正しくならないことに注意されたい.



また場合によっては,
脚注をつけた位置と脚注本体とを別の段に置く方がよいこともある.
この場合には,
\|\footnotemark| コマンドや \|\footnotetext| コマンドを使って対処していただきたい.


なお,脚注番号は論文内で通し番号で出力される.




%4.3.8
\subsubsection{OverfullとUnderfull}

組版時にはoverfullを起こさないことを原則としている.
従って,まず提出するソースが著者の環境でoverfullを起こさないように,
文章を工夫するなどの最善の努力を払っていただきたい.
但し,\|flushleft| 環境,\|\\|,\|\linebreak| などによる両端揃えをしない形でのoverfullの回避は,
できるだけ避けていただきたい.
また著者の執筆時点では発生しないoverfullが,
組版時の環境では発生することもある.
このような事態をできるだけ回避するために,
文中の長い数式や \|\verb| を避ける,
パラグラフの先頭付近では長い英単語を使用しない,
などの注意を払うようにして頂きたい.



%4.4
\subsection{数式}\label{sec:Item}

%4.4.1
\subsubsection{本文中の数式}

本文中の数式は \|$| と \|$|, \|\(| と \|\)|, あるいは \|math| 環境のいずれで囲んでもよい.



%4.4.2
\subsubsection{別組の数式}

別組数式(displayed math)については \|$$| と \|$$| は使用せずに,
\|\[| と \|\]| で囲むか,
\|displaymath|, \|equation|, \|eqnarray| のいずれかの環境を用いる.これらは
%
\begin{equation}
\Delta_l = \sum_{i=l|1}^L\delta_{pi}
\end{equation}
%
のように,センタリングではなく固定字下げで数式を出力し,
かつ背が高い数式による行送りの乱れを吸収する機能がある.




%4.4.3
\subsubsection{eqnarray環境}

互いに関連する別組の数式が2行以上連続して現れる場合には,
単に\|\[| と \|\]|,
あるいは \|\begin{equation}| と\|\end{equation}| で囲った数式を書き並べるのではなく,
\|\begin|\allowbreak\|{eqnarray}| と \|\end{eqnarray}| を使って,
等号(あるいは不等号)の位置で縦揃えを行なった方が読みやすい.



%4.4.4
\subsubsection{数式のフォント}


\LaTeX が標準的にサポートしているもの以外の特殊な数式用フォントは,
できるだけ使わないようにされたい.
どうしても使用しなければならない場合には,
その旨申し出て頂くとともに,
組版工程に深く関与して頂くこともあることに留意されたい.


\begin{figure}[tb]
\setbox0\vbox{
\hbox{\|\begin{figure}[tb]|}
\hbox{\quad \|<|図本体の指定\|>|}
\hbox{\|\caption{<|和文見出し\|>}|}
\hbox{\|\ecaption{<|英文見出し\|>}|}
\hbox{\|\label{| $\ldots$ \|}|}
\hbox{\|\end{figure}|}
}
\centerline{\fbox{\box0}}
\caption{1段幅の図}
\ecaption{Single column figure with caption\\
explicitly broken by $\backslash\backslash$.}
\label{fig:single}
\end{figure}


%4.5
\subsection{図}

1段の幅におさまる図は,
\figref{fig:single} の形式で指定する.
位置の指定に \|h| は使わない.
また,図の下に和文と英文の双方の見出しを,
\|\caption| と \|\ecaption| で指定する.
文字数が多い見出しはは自動的に改行して最大幅の行を基準にセンタリングするが,
見出しが2行になる場合には適宜 \|\\| を挿入して改行したほうが
良い結果となることがしばしばある(\figref{fig:single} の英文見出しを参照).
図の参照は \|\figref{<|ラベル\|>}| を用いて行なう.





また紙面スペースの節約のために,
1つの \|figure|(または \|table|)環境の中に複数の図表を並べて表示したい場合には,
\figref{fig:left} と \tabref{tab:right} のように個々の
図表と各々の \|\caption|/\|\ecaption| を \|minipage| 環境に入れることで実現できる.
なお図と表が混在する場合,
\|minipage| 環境の中で\|\CaptionType{figure}| あるいは \|\CaptionType| \|{table}| を指定すれば,
外側の環境が \|figure| であっても \|table| であっても指定された見出しが得られる.



\begin{figure}[tb]
\begin{minipage}[t]{0.5\columnwidth}
\footnotesize
\setbox0\vbox{
\hbox{\|\begin{minipage}[t]%|}
\hbox{\|  {0.5\columnwidth}|}
\hbox{\|\CaptionType{table}|}
\hbox{\|\caption{| \ldots \|}|}
\hbox{\|\ecaption{| \ldots \|}|}
\hbox{\|\label{| \ldots \|}|}
\hbox{\|\makebox[\textwidth][c]{%|}
\hbox{\|\begin{tabular}[t]{lcr}|}
\hbox{\|\hline\hline|}
\hbox{\|left&center&right\\\hline|}
\hbox{\|L1&C1&R1\\|}
\hbox{\|L2&C2&R2\\\hline|}
\hbox{\|\end{tabular}}|}
\hbox{\|\end{minipage}|}}
\hbox{}
\centerline{\fbox{\box0}}
\caption{\protect\tabref*{tab:right} の中身}
\ecaption{Contents of table \protect\ref{tab:right}.}
\label{fig:left}
\end{minipage}%
\begin{minipage}[t]{0.5\columnwidth}
\CaptionType{table}
\caption{\protect\figref*{fig:left} で作成した表}
\ecaption{A table built by\\ Fig.\,\protect\ref{fig:left}.}
\label{tab:right}
\vskip1mm
\makebox[\textwidth][c]{\begin{tabular}[t]{lcr}\hline\hline
left&center&right\\\hline
L1&C1&R1\\
L2&C2&R2\\\hline
\end{tabular}}
\end{minipage}
\end{figure}

\begin{figure*}[tb]
\setbox0\vbox{\large
\hbox{\|\begin{figure*}[t]|}
\hbox{\quad \|<|図本体の指定\|>|}
\hbox{\|\caption{<|和文見出し\|>}|}
\hbox{\|\ecaption{<|英文見出し\|>}|}
\hbox{\|\label{| $\ldots$ \|}|}
\hbox{\|\end{figure*}|}}
\centerline{\fbox{\hbox to.9\textwidth{\hss\box0\hss}}}
\caption{2段幅の図}
\ecaption{Double column figure.}
\label{fig:double}
%\vspace*{-2.5mm}
\end{figure*}


2段の幅にまたがる図は,
\figref{fig:double} の形式で指定する.
位置の指定は \|t| しか使えない.
図の中身では本文と違い,
どのような大きさのフォントを使用しても構わない(\figref{fig:double} 参照).
また図の中身として,
encapsulate されたPostScriptファイル(いわゆるEPSファイル)を読み込むこともできる.
読み込みのためには,プリアンブルで
%
\begin{quote}
\|\usepackage{graphicx}|
\end{quote}
%
を行った上で,
\|\includegraphics| コマンドを図を埋め込む箇所に置き,
その引数にファイル名(など)を指定する.




%4.6
\subsection{表}

表の罫線はなるべく少なくするのが,
仕上がりをすっきりさせるコツである.
罫線をつける場合には,一番上の罫線には二重線を使い,
左右の端には縦の罫線をつけない (\tabref{tab:example}).
表中のフォントサイズのデフォルトは\|\footnotesize|である.


また,表の上に和文と英文の双方の見出しを,
 \|\caption|と \|\ecaption| で指定する.
表の参照は \|\tabref{<|ラベル\|>}| を用いて行なう.



\begin{table}[tb]
\caption{表の例}
\ecaption{An example of table.}
\label{tab:example}
\hbox to\hsize{\hfil
\begin{tabular}{l|lll}\hline\hline
& column1 & column2 & column3 \\\hline
row1 &	item 1,1 & item 2,1 & ---\\
row2 &	---      & item 2,2 & item 3,2 \\
row3 &	item 1,3 & item 2,3 & item 3,3 \\
row4 &	item 1,4 & item 2,4 & item 3,4 \\\hline
\end{tabular}\hfil}
\end{table}



%4.7
\subsection{参考文献・謝辞}

%4.7.1
\subsubsection{参考文献の参照}

本文中で参考文献を参照する場合には\|\cite|を使用する.
参照されたラベルは自動的にソートされ,
\|[]|でそれぞれ区切られる.
%
\begin{quote}
文献 \|\cite{companion,okumura}| は \LaTeX の総合的な解説書である.
\end{quote}
%
と書くと;
%
\begin{quote}
文献\cite{companion,okumura}は \LaTeX の総合的な解説書である.
\end{quote}
%
が得られる.

%4.7.2
\subsubsection{参考文献リスト}
参考文献リストには,
原則として本文中で引用した文献のみを列挙する.
順序は参照順あるいは第一著者の苗字のアルファベット順とする.
文献リストはBiB\TeX と\verb+ipsjunsrt.bst+(参照順)
または\verb+ipsjsort.bst+(アルファベット順)を用いて作り,
\verb+\bibliograhpystyle+と\verb+\bibliography+コマンドにより
利用することが出来る.
これらを用いれば,
規定の体裁にあったものができるので,
できるだけ利用していただきたい.
また製版用のファイル群には\verb+.bib+ファイルではなく\verb+.bbl+ファイルを
必ず含めることに注意されたい.
一方,何らかの理由でthebibliography環境で文献リストを
「手作り」しなければならない場合は,
このガイドの参考文献リストを注意深く見て,
そのスタイルにしたがっていただきたい.



%4.7.3
\subsubsection{謝辞}

謝辞がある場合には,参考文献リストの直前に置き,\|acknowledgment|環境の中に入れる.


%4.8
\subsection{著者紹介}



本文の最後(\|\end{document}| の直前)に,以下のように著者紹介を記述する.
\begin{quote}
\|\begin{biography}|\\
\|\profile{m}{<|第一著者名\|>}{|第一著者の紹介\|}|\\
\|\profile{m,F}{<|第二著者名\|>}{|第二著者の紹介\|}|\\
\|\profile{m}{<|$\dots$\|>}{|$ldots$\|}|\\
\|\end{biography}|
\end{quote}
なお最初の引数を変えることで,会員種別が変わる.
\begin{quote}
\|名誉会員   :h|\\
\|正会員    :m|\\
\|学生会員   :s|\\
\|ジュニア会員 :j|\\
\|非会員    :n|
\end{quote}
また会員種別と同時に,称号を表記することもできる.
\begin{quote}
\|フェロー   :F|\\
\|シニア会員  :E|\\
\|終身会員   :L|
\end{quote}
なお称号は著者紹介の末尾に表記される.


著者紹介用の写真は縦30ミリ×横25ミリのサイズにて使用する.
頭の一部が切れているものや背景と顔の輪郭が区別しにくいものなどは避け,
背景は無いもの,または薄い色のものを使用するのが望ましい.
なお写真データは,解像度300dpi以上,100万画素以上のカメラを使用したデータを推奨する.
電子データを用意できない場合は,証明写真を送付されたい.
また,著者紹介用写真は組版を行う際に印刷業者で取り込むため,
原稿作成時に写真を取り込む必要はない.
