\section{投稿の流れ}

%2.1
\subsection{準備}

情報処理学会論文誌ジャーナルの \LaTeX スタイルファイルを含む論文執筆キッ
トは
\begin{quote}
\small
\|http://www.ipsj.or.jp/jip/submit/style.html|
\end{quote}
からダウンロードすることができる.論文執筆キットは以下のファイルを含んで
いる.
\begin{enumerate}
\item \|ipsj.cls      |: 最終原稿用スタイルファイル
\item \|ipsjdraft.sty |: 投稿用スタイル(査読用)
\item \|ipsjpref.sty  |: 序文用スタイル
\item \|jsample.tex   |: 本稿のソースファイル
\item \|esample.tex   |: 英文サンプルのソースファイル
\item \|ipsjsort.bst  |: jBibTEX スタイル(著者名順)
\item \|ipsjunsrt.bst |: jBibTEX スタイル(出現順)
\item \|bibsample.bib |: 文献リストのサンプル
\item \|ebibsample.bib|: 英文文献リストのサンプル
\end{enumerate}
キットはUnix用,Windows (DOS)用,Macintosh用などが用意されており,著者の
作業環境に応じたものを選択できるようになっている.また,実行環境としては
\LaTeXe を前提としているので,準備されたい.

Microsoft Wordに関しては,投稿されたフォーマットを基に,業者が \LaTeX に
変換して組版を行うので,あくまでも参考としてしか使わないことを承知して頂
きたい.

%2.2
\subsection{最終原稿の作成と投稿}

本稿に従って用意した投稿用原稿の \LaTeX ソースからpdfファイルを作成し,
Adobeのpdf readerで読めることを確認した後,
\begin{quote}
\small
\|https://www.ipsj.or.jp/prms/author_pre_submit.do|
\end{quote}
のPRMS (Paper Review Management System)にメールアドレスを登録し,送られ
たきたメールに従って,指定されたURLから投稿する.投稿の流れについては,
\begin{quote}
\small
\|http://www.ipsj.or.jp/journal/submit/manual/|
\|manual_j_for_Author.pdf|
\end{quote}
を参照されたい.

なお,情報処理学会論文誌ジャーナルでは,論文の著者が査読者の名前を知るこ
とがないだけなく,査読者も著者の名前を知らないダブルブラインドの査読を取
り入れている.このため,投稿版では,原稿に著者名とその所属は表示しないよ
うにする必要がある.

%2.3
\subsection{最終原稿の作成とファイルの送付}

投稿した論文の採録が決定したら,査読者からのコメントなどにしたがって原稿
を修正し,著者紹介など投稿時になかった項目があれば追加する.また図表など
のレイアウトも最終的なものとする.なお後の校正の手間を最小にするために,
この段階で記述の誤りなどを完全に除去するように綿密にチェックして頂きたい.

最終版では,著者名およびその所属を表示すると同時に,学会より指示された巻
数,号数,先頭ページ番号,受付/採録年月日(年は西暦)を記述する.なお学
会からの指示がない項目に関しては,記述しなくてよい.

学会へは{\bf \LaTeX ファイル(をまとめたもの)とハードコピーの双方を}送
付する.送付するファイル群の標準的な構成は \|.tex| と \|.bbl| であり,こ
の他にPostScriptファイルや特別なスタイルファイルがあれば付加する.なお
\|.tex| は印刷業者が修正することがあるので,{必ず一つのファイルにする}.
また必要なファイルが全てそろっていること,特に特別なスタイルファイルに洩
れがないことを,注意深く確認して頂きたい.

ファイルの送付方法などについては,採録通知とともに学会事務局から送られる
指示に従う.

%2.4
\subsection{著者校正・組版・出版}

学会では用語や用字を一定の基準に従って修正することがある.また \LaTeX の
実行環境の差異などによって著者が作成したハードコピーと実際の組版結果が微
妙に異なることがある.これらの修正や差異が問題ないかを最終的に確認するた
めに,著者にゲラ刷りが送られるので,もし問題があれば朱書によって指摘して
返送する.なお{\bf この段階での記述誤りの修正は原則として認められない}の
で,原稿送付時に細心の注意を払っていただきたい.

その後,著者の校正に基づき最終的な組版を行ない,オンライン出版する.



