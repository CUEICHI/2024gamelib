\section{論文内容に関する指針}

論文の内容について,論文誌ジャーナル編集委員会で作成した「べからず集」を
以下に示す.投稿前のチェックリストとして利用頂きたい.これ以外にも,査読
者用,メタ査読者用の「べからず集」\cite{webpage2}も公開しているので,参
照されたい.また,作文技術に関する \cite{book1, book2, book3, book4}のよ
うな書籍も参考になる.

%5.1
\subsection{書き方の基本}

\begin{itemize}
 \item[$\Box$] 研究の新規性,有用性,信頼性が読者に伝わるように記述する.
 \item[$\Box$] 読み手に,読みやすい文章を心がける(内容が前後する,背景・
	       課題の設定が不明瞭などは読者にとって負担).
 \item[$\Box$] 解決すべき問題が汎用化(一般的に記述)されていないのは再
	       考を要する(XX大学の問題という記述に終始).あるいは,
	       (単に「作りました」だけで)解決すべき問題そのものの記述
	       がないのは再考を要する.
 \item[$\Box$] 結論が明確に記されていない,または,範囲,限界,問題点な
	       どの指摘が適切ではない,または,結論が内容にそったもので
	       はないものは再考を要する.
 \item[$\Box$] 科学技術論文として不適当な表現や,分かりにくい表現がある
	       のは再考を要する.
 \item[$\Box$] 極端な口語体や,長文の連続などは再考を要する.
 \item[$\Box$] 章,節のたて方,全体の構成等が適切でない文章は再考を要す
	       る.
 \item[$\Box$] 文中の文脈から推測しないと内容の把握が困難な論文にしない.
 \item[$\Box$] 説明に飛躍した点があり,仮説等の説明が十分ではないのは再
	       考を要する.
 \item[$\Box$] 説明に冗長な点,逆に簡単すぎる点があるのは再考を要する.
 \item[$\Box$] 未定義語を減らす.
\end{itemize}


%5.2
\subsection{新規性と有効性を明確に示す}

\begin{itemize}
 \item[$\Box$] 在来研究との関連,研究の動機,ねらい等が明確に説明されて
	       いないのは再考を要する.
 \item[$\Box$] 既知/公知の技術が何であって,何を新しいアイデアとして提
	       案しているのかが書かれていないのは再考を要する.
 \item[$\Box$] 十分な参考文献は新規性の主張に欠かせない.
 \item[$\Box$] 提案内容の説明が,概念的または抽象的な水準に終始していて,
	       読者が提案内容を理解できない(それだけで新規性が感じられ
	       ないもの)のは再考を要する.
 \item[$\Box$] 論文で提案した方法の有効性の主張がない,またはきわめて貧
	       弱なのは再考を要する.
\end{itemize}

%5.3
\subsection{書き方に関する具体的な注意}

\begin{itemize}
 \item[$\Box$] 和文標題が内容を適切に表現していないのは再考を要する.
 \item[$\Box$] 英文標題が内容を適切に表現していない,または英語として適
	       切でないのは再考を要する.
 \item[$\Box$] アブストラクトが主旨を適切に表現していない,または英文が
	       適切ではないのは再考を要する.
 \item[$\Box$] 記号・略号等が周知のものでなく,または,用語が適切でなく,
	       または,図・表の説明が適当ではないのは再考を要する.
 \item[$\Box$] 個人的あるいは非常に小さなグループ/企業だけで通用するよ
	       うな用語が特別な説明もなしに多用されているのは再考を要す
	       る.
 \item[$\Box$] 図表自体は十分に明確ではない,または誤りがあるのは再考を
	       要する.
 \item[$\Box$] 図表が鮮明ではないのは再考を要する.
 \item[$\Box$] 図表が大きさ,縮尺の指定が適切でないのは再考を要する.
\end{itemize}

%5.4
\subsection{参考文献}

\begin{itemize}
 \item[$\Box$] 参考文献は10件以上必要(分野によっては20件以上,30件以上
	       という意見もある).
 \item[$\Box$] 十分な参考文献は新規性の主張に欠かせない.
 \item[$\Box$] 適切な文献が引用されておらず,その数も適切ではないのは再
	       考を要する.
 \item[$\Box$] 日本人によるしかるべき論文を引用することで日本人研究コミュ
	       ニティの発展につながる.
 \item[$\Box$] 参考文献は自分のものばかりではだめ.
\end{itemize}

%5.5
\subsection{二重投稿}

\begin{itemize}
 \item[$\Box$] 二重投稿はしてはならない ─ ただし国際会議に採択された論
	       文を著作権が問題にならないように投稿することは構わない.
 \item[$\Box$] 他の論文とまったく同じ図表を引用の明示なしに利用すること
	       は禁止.
 \item[$\Box$] 既発表の論文等との間に重複があるのは再考を要する.
\end{itemize}

%5.6
\subsection{他の人に読んでもらう}

\begin{itemize}
 \item[$\Box$] 投稿経験が少ない人は,採録された経験の豊富な人に校正して
	       もらう.
 \item[$\Box$] 読者の立場から見て論理的な飛躍がないかに注意して記述する.
\end{itemize}

%5.7
\subsection{その他}

\begin{itemize}
 \item[$\Box$] 条件付採録後の修正で,採録条件以外を理由もなく修正するこ
	       とは禁止.
 \item[$\Box$] ダブルブラインドなので査読者は選べない.
 \item[$\Box$] 投稿前にチェックリストの各項目を満たしているか,必ず確認
	       する.
\end{itemize}

%6
\section{おわりに}

本稿では,A4縦型2段組み用に変更したスタイルファイルを用いた論文のフォー
マット方法と,論文誌ジャーナル編集委員会がまとめた「べからず集」に基づく
論文の書き方を示した.内容的にまだ不十分の部分が多いため,意見,要望等を
\begin{quote}
 \|editt@ipsj.or.jp|
\end{quote}
までお寄せ頂きたい.
